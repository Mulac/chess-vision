\chapter{Methods}
\label{chapter2}

<Everything that comes under the `Methods' criterion in the mark scheme should be described in one, or possibly more than one, chapter(s).>

\section{Data Collection}
At the heart of any machine learning project is the data.  
It is as important, often more important, than the code and presents many interesting
challenges.  **Why is this?**
Discussed in the following sections are some of the challenges and decisions that were considered.

\subsection{Sensors}
The eminent challenge is acquiring data in the first place.  This is highly context 
dependant, but as vision is primarily focused on spatial awareness the discussion 
will be limited to the sensors that can measure it.

Sensor choice is an important choice for any robotics application as there are 
important tradeoffs, as with any engineering challenge, which must be considered.

**Outline some of the tradeoffs between spatial sensors**

One important distinction to make is the difference between training and inference.
Requirements at the time of training my differ significantly to the requirements at 
inference.  Processing power, energy supply and realtime operation are some
of the constraints that will have to be met when considering different sensors.

The sensor used throughout this project is the RealSense SR305 which is a RGB-D camera
using structured and coded light to determine depth, it functions best indoors or in a 
controlled lighting situation.  For the reasons outlined above the RGB camera stream is 
mainly relied upon but there will be some discussion and comparision of piece detection 
with the depth sensor.  

\subsection{Auto-Labelling}
A closely related challenge of acquiring the data is that of labelling it too.
It is widely known that neural networks scale with the number of examples \cite{}.  
This will be explicitly explored for Chess Vision in section 3.1.
This however poses the question about how do we get access to a lot of labelled data 
for chess.  **Some examples of other auto labelling techniques**
The overall approach I took...

\subsection{Dataset Versioning}
With all this data the next challenge becomes self evident.
It is concerned with the question: How do we manage all of this?
Some of the problems... and why you need versioning...
Transitioning from git lfs to aws s3.  Perhaps a quick mention of other solutions.
How the Game class solves some of these challenges for us.


\section{Model Training}
\lipsum[7]

\subsection{Architectures}
\lipsum[6]

\subsection{Experiment Tracking}
\lipsum[6]



\section{Realtime Inference}
\lipsum[7]

\subsection{The Realiable Approach}
\lipsum[6]

\subsection{The Not-So-Reliable Approach}
\lipsum[6]
