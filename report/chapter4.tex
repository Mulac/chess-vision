\chapter{Discussion}
\label{conclusions}

\section{Workflow}
Throughout development, every change was monitored and evaluated.  As expressed in \nameref{methods}, the use of the Game and Labeller
abstractions along with guild for experiment tracking greatly increased speed and ability of evaluate changes. 

\section{Conclusions}

\subsection{Piece Recognition}

\subsection{Dataset Management}

\section{Ideas for future work}
Firstly it would be nice to explore these methods with more extreme camera angles.  This would probably include extending the labeller too add more 
margin in one direction to account for the perspective. If assumptions about the environment are to be kept minimal
then localising pieces in 3 dimensional space should be a requirement for robotic manipulation to be possible.

It is the author's belief that "3d reconstruction" would kill two birds with one stone and the direction that future research should focus.

Explain.  Similar work in other areas.