\chapter{Discussion}
\label{conclusions}

\section{Workflow}
Throughout development, every change was monitored and evaluated.  As expressed in \nameref{methods}, the use of the Game and Labeller
abstractions along with guild for experiment tracking greatly increased speed and ability of evaluate changes.  

What was the result of the data labelling pipeline
How well did it work
What could be improved

\subsection{Dataset Management}


\section{Board Segmentation}
The decision to use aruco markers, while impractical for release, aided development in a couple of ways.  Firstly, they're reliable \cite{}.
To correctly classify pieces, the segmentation of board squares was critical as a small error of even 10mm could completely throw the
model off as due to added perspective shift a square could now contain two pieces.  Compared to the iterative heatmap method \cite{} aruco 
marker are not the best, but in a controlled environment this did not become a problem and more importantly they worked even with pieces on the
board which enabled recalibration during inference and while collecting data.  This is unlike a lot of other proposed solutions \cite{}.
Secondly they're fast \cite{}.  The iterative heatmap method can take around 5 seconds to segment the board which while not terrible can add up when 
relabelling lots of data for many model experiments.

As this project focused on piece recognition and the development of an inference system, aruco markers proved to be a sensible choice.  However, 
they come with deal breaking consequences if this method is to be used in production with many boards.  It's just too impractical to print and fix
markers every time a new board is to be used with the system and again goes directly against the main purpose of this project: autonomy.


\section{Inference}
Added the ability to take snapshots of misclassification during inference which could be used to fine-tune the model and leads us down the 
path of continual learning.  Outside the scope of this project.

\subsection{Multitask}
Initially a benefit of having the depth sensor is an easier way to detech piece presence.  
Fixed threshold vs clustering.  Adding a margin.  How we actually did it.
Using paried T-test to evaulate.

\section{Dataset}
I found not only managing datasets but choosing and building them to be a big yet fun task.  No doubt by the legends who've built tools for me to use,
my experience has been that most of the work is in the data. Creative ways to collect it, label it and fit it to pre-existing network architectures.  

It was not until I recorded and labeled lots of data (1000s images) was it until I suddenly started to see new patterns I had not seen before.  Small 
things like dark shadows, motion blur, pieces half across squares, poor resolution or exposured frames.

More control over the sensor hardware sensors may have been useful.  Changes in the way we generate image (as in the autolabeller) or
adding augmentation such as cropping and blurring helped which these changes that I would not have seen in my metrics before hand.

\section{Conclusion}



\section{Ideas for future work}
Firstly it would be nice to explore these methods with more extreme camera angles.  This would probably include extending the labeller too add more 
margin in one direction to account for the perspective. If assumptions about the environment are to be kept minimal
then localising pieces in 3 dimensional space should be a requirement for robotic manipulation to be possible.

Perhaps worth taking a step back and reconsidering the unanimously made decision by which chessboard state recognition is done.  Instead of splitting 
the board up into squares and using simple image classification it would be more useful for robotic systems to have a 3D representation of the space.
It is the author's belief that "3d reconstruction" would kill two birds with one stone and the direction that future research should focus. 

Explain.  Similar work in other areas.