\chapter{Introduction and Background Research}

% You can cite chapters by using '\ref{chapter1}', where the label must
% match that given in the 'label' command, as on the next line.
\label{chapter1}

% Sections and sub-sections can be declared using \section and \subsection.
% There is also a \subsubsection, but consider carefully if you really need
% so many layers of section structure.
\section{Introduction}

Algorithms such as Deep Blue \cite{parikh1980adaptive}, AlphaZero \cite{} and more recently Player of Games\cite{}
have enabled computers to out smart the smartest humans at the game of Chess.
But all these algorithms are bound to the digital world, rendered useless when
competing against humans on a real board. This project aims to explore a major
component of this: vision.
Consider the vision problem for chess to be two-fold: what is the current board
state and where are all of the pieces?  With this information, in combination with the
previous algorithms and a robot arm, the computer is no longer bound to the
digital world.
In particular this project's will focus on the former, that is, to produce and present a solution for determining the
state of a chess board from a video stream.  A solution reliable enough to live up to the likes of AlphaZero 
in a robotic solution.  There will be a focus on deep learning techniques, with
consideration for best practice in machine learning operations.  There is also the aim to share the 
tools to more easily manage and create new datasets in this area of chessboard state recognition.  Something 
called for by \cite{} as a serious challenge and priority for future research.

% Must provide evidence of a literature review. Use sections
% and subsections as they make sense for your project.
\section{Literature review}
\label{research}
For humans the hard part of chess is planning, this is not the case for computers, instead 
recognition and localisation of objects in 3D space, along with manipulation, present much greater
challenges. \cite{}

Make reference to humans huge allocation of resources to vision. \cite{}
Why is it so hard for computers then? It's an inverse problem. 
Compare to solving the decision problem (minimax).  
The statistal calculation of whether to trade Queen's or block with a pawn has now become trival.  

\subsection{A Short History of Computer Vision}
\subsubsection{Classical Techniques}
How we can use classical techniques to understand images.  Neural networks have taken over almost all of the heavy lifting for high level inference.
Give an indication of time scale here.
\subsubsection{Image Recognition}
the way back in 1980 \cite{} convolutions showed promise in simple computer vision tasks, 
convolutions since have showed extreme promise \cite{}
Le Ye Cunn's work with convolutions \cite{} and MNIST \cite{} and more recently ImageNet \cite{} having become incredibly well known.
New methods such as Transformers from the world have NLP have generalised the convolution operation have proved 
very successful and lots of work here has been applied to vision. \cite{} \cite{} \cite{} (Attention is all you need, ViT, generic model from deepmind)
\subsubsection{Object Detection}
But in most applications there will be many things we want to recognise in an image.  The RCNN \cite{} enters.  How Faster-RCNN improves on this \cite{}.
Why YOLO \cite{} has been so successful (realtime).  Transformers are applicable here too.
\subsubsection{Instance Segmentation}
Why bounding boxes are not enough.  What is segmentation? Why instance segmentation is what we actually want. \cite{}
\subsubsection{Adding More Dimensions}
The real world is not percepived in static 2d images.  How do we add an understanding of 3 dimensions and time in our computer vision models? \cite{}
Important for localisation in the real world.  Important for understanding things like object permenance.

\subsection{Tools}
\subsubsection{Dataset Collection??}
\subsubsection{Chess Engine??}
\subsubsection{Deep Learning Library??}

\subsection{Computer Vision for Chess}
Despite chess being a very narrow application of computer vision, the amount of research effort gone into the problem of determining 
board state is not insignificant. 
A variety of approaches have been tested and explored for which the following section will attempt to fairly summarise.

As in \cite{} we will futher split the vision problem into two further problems for analysis: board detection and piece recognition.

\subsubsection{Board Detection}
The problem of board detection is not specific to chess but also finds heavy research from areas such as camera calibration \cite{}. 
The built in camera calibration functions in opencv \cite{} and matlab \cite{} are used in many previous works \cite{} which 
provide a quick  and precise solution for board detection but becomes unusable when any pieces are present on the board forcing those authors to 
take the approach of an initial setup stage at inference making the solution unfit to changes in board position during inference. 

Due to a chessboard's simple features many early works of line and corner point detection can be applied.  For example Hough 
transforms \cite{} are used to detect the lines of a chessboard \cite{}.  Corner point detection methods such as the 
Harris and Stephens's \cite{} were also common among solutions \cite{}, with some authors combining approaches with further 
processing such as canny edge detection \cite{} to yield more reliable results.

The ChESS was another corner detection algorithm that out performed the Harris and Stephen's algorithm \cite{} which was 
interestingly creating for real-time measurement of lung function in humans, further demonstrating the general applicability 
of chess board detection.

There are many other algorithms that require simplifications such as green-red chessboards \cite{}, multiple camera angles \cite{}, 
or even user input for entering the corners of the chessboard \cite{}.

The most impressive work came out of Poznan University of Technology which proposes many interesting ideas that performs more 
reliably in a wider range of difficult situation such as pieces being present on the board \cite{}.  The employ an iterative 
approach with each iteration containing 3 sub-tasks: line segment detection, lattice point search and chessboard position search.
In each iteration of line detection a canny lines detector \cite{} is used on many preprocessed variations of the input image 
to maximize line segment detections which are then merged using a novel linking function. The lattice point search starts with 
the intersection of each of the merged lines as input, converts these to a 21x21 pixel binary image of the surrounding area and 
runs them through a classifier to remove outliers.  The addition of a neural network as a classifier is what greatly improves 
the generality of the proposed solution as it can be resistant to lattice points that are perhaps partially covered by a chess piece.
The final sub-task then calculates the predicted corner points of the chessboard which minimizes the area will maximizing the 
likely hood that the chessboard is fully contained.  It is this area that is then used as input to the next iteration until the 
points converge.  

\subsubsection{Piece Recognition}

Go over the main appraoches with references.

Go to standford dude and the heatmap guys 
as the best approach out there. They use SIFT and hard coded color alogrithms.  heatmap guys improved on this 
only by adding more restrictions by assuming the board much be valid and making statistical assumptions on 
what state is most likely.

Our approach will not have such restrictions and will leverage neural networks.

Some others have attempted this before but failed for these reasons...


\subsection{Prior Work From the Author}
Mention robotic arm for two counter board games and automatic differentiation library.
