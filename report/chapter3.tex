\chapter{Results}
\label{results}

\section{Model Evaluation}
Throughout development, every change was monitored and evaluated.  As expressed in \nameref{methods} some particular design decisions
and tools greatly helped at making that possible.

\subsection{Board Segmentation}
The decision to use aruco markers, while impractical for release, aided development in a couple of ways.  Firstly, they're reliable \cite{}.
To correctly classify pieces, the segmentation of board squares was critical as a small error of even 10mm could completely throw the
model off as due to added perspective shift a square could now contain two pieces.  Compared to the iterative heatmap method \cite{} aruco 
marker are not the best, but in a controlled environment this did not become a problem and more importantly they worked even with pieces on the
board which enabled recalibration during inference and while collecting data.  This is unlike a lot of other proposed solutions \cite{}.
Secondly they're fast \cite{}.  The iterative heatmap method can take around 5 seconds to segment the board which while not terrible can add up when 
relabelling lots of data for many model experiments.

As this project focused on piece recognition and the development of an inference system, aruco markers proved to be a sensible choice.  However, 
they come with deal breaking consequences if this method is to be used in production with many boards.  It's just too impractical to print and fix
markers every time a new board is to be used with the system and again goes directly against the main purpose of this project: autonomy.

Time comparison for the 3.  And compare accuracy of heatmap against aruco.

\subsection{Multitask Learning}
How does it compare?

\subsection{Piece Recognition}
Use top-1 and top-5 to measure performance of different models.
Will include basic evaluation.  What happens what you increase layers, use more data, data augmentation.
Include Recall / Specificity / Sensitivity

A benefit of having the depth sensor is an easier way to detech piece presence.  
Fixed threshold vs clustering.  Adding a margin.  How we actually did it.
Using paried T-test to evaulate.

Using a neural network instead as an additional class with our piece recognition network.
Compare that two having a two stage network, the first for piece detection and the second for recognition.


\subsection{Hyperparameter Tuning}
Augmentation as well?
Talk about grayscale (a lot of previous works used gray scale do performant CNNs aid from gray or hinder?)

\subsection{Deep Dive into CNNs}
Visualising convolutional layers to analyse effectiveness.

\section{Realtime Analysis}
Frames per second.  Problems.

\subsection{Trials}
table of end-to-end experiments

\section{Comparison to Existing Solutions}
\begin{center}
\begin{tabular}{|c|c|}
    \multicolumn{2}{c}{Chessboard 1} \\
    \hline
    Method & Accuracy \\
    \hline
    proposed & 94\% \\
    \cite{} & 78\% \\
    \cite{} & 65\%  \\
    \hline
\end{tabular}
\end{center}

Talk about speed of inference and any other limitations of both presented and other work.